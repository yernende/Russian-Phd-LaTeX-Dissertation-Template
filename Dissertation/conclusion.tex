\chapter*{ЗАКЛЮЧЕНИЕ}                       % Заголовок
\addcontentsline{toc}{chapter}{ЗАКЛЮЧЕНИЕ}  % Добавляем его в оглавление

В результате данной работы разработан комплекс программных средств, пригодных для реализации методов присоединённых
плоских волн и рассеяных волн для изучения электронных состояний и физических свойств молекул и кластеров произвольной
геометрии, в том числе с применением средств распараллеливания вычислений. Кроме того, было соблюдено требование
использования только свободных программных библиотек и пакетов, что позволяет в перспективе опубликовать программы
под такой же свободной лицензией.

Проведены расчёты методом присоединённых плоских волн для некоторых молекул и кластеров, и в них обнаружен магнитный
момент.

В будущем авторы планируют масштабировать созданные программы на случай больших (до ста атомов) молекул и кластеров,
в частности, с использованием средств сетевого распараллеливания. По результатам проведённых вычислений готовятся
статьи.

%% Согласно ГОСТ Р 7.0.11-2011:
%% 5.3.3 В заключении диссертации излагают итоги выполненного исследования, рекомендации, перспективы дальнейшей разработки темы.
%% 9.2.3 В заключении автореферата диссертации излагают итоги данного исследования, рекомендации и перспективы дальнейшей разработки темы.
%% Поэтому имеет смысл сделать эту часть общей и загрузить из одного файла в автореферат и в диссертацию:

% Основные результаты работы заключаются в следующем.
% \input{common/concl}
% И какая-нибудь заключающая фраза.
%
% Последний параграф может включать благодарности.  В заключение автор
% выражает благодарность и большую признательность научному руководителю
% Иванову~И.\,И. за поддержку, помощь, обсуждение результатов и~научное
% руководство. Также автор благодарит Сидорова~А.\,А. и~Петрова~Б.\,Б.
% за помощь в~работе с~образцами, Рабиновича~В.\,В. за предоставленные
% образцы и~обсуждение результатов, Занудятину~Г.\,Г. и авторов шаблона
% *Russian-Phd-LaTeX-Dissertation-Template* за~помощь в оформлении
% диссертации. Автор также благодарит много разных людей
% и~всех, кто сделал настоящую работу автора возможной.

% \FloatBarrier
% \chapter*{АННОТАЦИЯ}
% Целью данной работы явилась разработка и практическое использование комплекса программ для исследования электронных
% состояний и магнитных свойств молекул и кластеров произвольной геометрии с использованием методов, разработанных в
% теории твёрдых тел и показавших свою эффективность в данной области, а именно: метода присоединённых плоских волн и
% метода рассеянных волн. С использованием одних только свободных программных библиотек и пакетов, объединяя
% быстродействие языка программирования C++ и удобство я зыка программирования Python разработаны программы, способные
% считать энергетические спектры и волновые функции атомных систем произвольной структуры, для этого, в частности,
% понадобилось реализовать решение дифференциальных уравнений особого вида: уравнений Хартри"--~Фока "---для этого был создана
% программа, реализующая метод Нумерова решения обыкновенных дифференциальных уравнений второго порядка с отсутствующей
% первой производной. Рассмотрен ряд обменно"~корреляционных потенциалов, которые могут быть использовании при записи этих
% уравнений, практически задействован обменно"~корреляционный потенциал Гуннарссона"--~Лундквиста, поскольку он хорошо
% подходит для описания магнитных свойств. Создан программный код, позволяющий изображать трёхмерные плотности
% электронных состояний. Реализовано распараллеливание проводимых вычислений по различным потокам операционной системы.
% Реализовано вычисление некоторых специальных функций, отсутствующих в используемых математических пакетах. Проведены
% расчёты методом рассеянных волн для кластеров Al4C6, Ti6Ni4, Fe10 и Fe12, для всех было показано наличие магнетизма,
% посчитан магнитный момент, построены графики распределения плотности состояний электронов по энергиям. В концы работы,
% в приложении, приведены наиболее значимые отрывки из исходного кода, созданного в рамках данной работы.
