\chapter*{Введение}                         % Заголовок
\addcontentsline{toc}{chapter}{Введение}    % Добавляем его в оглавление

% \newcommand{\actuality}{}
% \newcommand{\progress}{}
% \newcommand{\aim}{{\textbf\aimTXT}}
% \newcommand{\tasks}{\textbf{\tasksTXT}}
% \newcommand{\novelty}{\textbf{\noveltyTXT}}
% \newcommand{\influence}{\textbf{\influenceTXT}}
% \newcommand{\methods}{\textbf{\methodsTXT}}
% \newcommand{\defpositions}{\textbf{\defpositionsTXT}}
% \newcommand{\reliability}{\textbf{\reliabilityTXT}}
% \newcommand{\probation}{\textbf{\probationTXT}}
% \newcommand{\contribution}{\textbf{\contributionTXT}}
% \newcommand{\publications}{\textbf{\publicationsTXT}}

% \input{common/characteristic} % Характеристика работы по структуре во введении и в автореферате не отличается (ГОСТ Р 7.0.11, пункты 5.3.1 и 9.2.1), потому её загружаем из одного и того же внешнего файла, предварительно задав форму выделения некоторым параметрам

Различные исследования\cite{localNerds1}\cite{localNerds2}\cite{SUNDARESAN200996}\cite{Venkatesan2004} показали
появление магнитных свойств в некоторых соединениях при деформации и охлаждении, ровно как и наличие магнитных свойств
у нанокластеров тех соединений, которые в макроскопическом масштабе немагнитны.

Настоящая работа состоит в разработке комплекса программ для исследования электронных состояний и магнитных состояний
молекул и кластеров. Для реализации этой цели предлагается использовать методы, разработанные в теории твёрдых тел
и показавшие свою эффективность в данной области.

Одним из таких методов является метод рассеянных волн\cite{sw-method}. Его применимость как для
простых молекул, так и для многоатомных кластеров и наноструктур была показана
ранее\cite{nyavro}. Есть в нём, однако, труднопреодолимый недостаток: повышение точности описания поведения
электронов в пространстве между атомами в рамках этого метода не представляется
возможным, что затрудняет его применение для пространственно разделённых молекул
и кластеров. Для того, чтобы избавиться от этого недостатка, в курсовой работе
начата реализация ещё одного из методов зонной теории кристаллов "---метода
присоединённых плоских волн\cite{afw-method}, в котором достаточно просто увеличить число
базисных функций "---плоских волн.

Ещё одна проблема, с которой сталкиваются при квантово"=механических расчётах
молекулярных систем "---это расчёт обменно"=кореляционных интегралов. Она
возникает из"=за нелокальности обменно"=корреляционного потенциала, что приводит
к значительному увеличению объёма вычислений из"=за необходимости расчёта
большого числа многоцентровых интегралов. Однако, если использовать локальные
аппроксимации обменного потенциала, то объём расчётов удаётся уменьшить на два
порядка. Такой подход был разработан в теории твёрдых тел и начало ему положено
в модели Слэтера, предложенной в середине прошлого века.

По мере использования различных способов расчёта обменно"=корреляционных
потенциалов предполагается тестировать их на простых объектах и использовать в
дальнейшей работе.

Следует отметить, что в работе не задействованы методы теории групп, поэтому сохраняется возможность изучения свойств
молекул и кластеров произвольной геометрии, а также изучать поведение систем в динамике, например, при деформациях
или изменениях температуры.

Несмотря на то, что в данной работе демонстрируется приложение разработанных программ на примере изучения магнитных
свойств кластеров, в перспективе их возможности на этом не ограничиваются. Их также можно использовать для вычисления
спектров, оптимизации геометрии и изучения прочих физических свойств молекул и кластеров.

Одним из принципов при разработке программного комплекса было ограничение списка задействованных программных
библиотек и пакетов свободным программным обеспечением, что увеличивает научную ценность работы, облегчая её
воспроизводимость, а также позволяет в перспективе опубликовать созданные программы под такой же свободной лицензией.

Работа состоит из обзора литературы, в который был также включен обзор использованных программных библиотек и пакетов,
за ним следует теоретический обзор использованных физических моделей, далее "---описание и обоснование принятых решений
при разработке программ, и, наконец, полученные результаты.
